\documentclass{article}

\usepackage[russian]{babel}
\usepackage[T2A]{fontenc}
\usepackage[utf8]{inputenc}
\usepackage{authblk}
\usepackage[dvipsnames]{xcolor}
\usepackage[T1]{fontenc}

\title{Code Style}
\author{Кулебакин Е.Д., ИА-031}
\affil{email: geka\_kule@mail.ru,  github: @Jencik4}
\date{Февраль 2022}

\begin{document}

\maketitle

\section{Введение}
Стиль кода (code style) — это набор правил для написания проекта. Единый стиль упрощает чтение, редактирование и написание кода.

\section{C++\cite{CPP}}
\subsection{Отступы\cite{one}}

Каждая новая инициализация размещается на новой строке.\\

\newcommand*{\Tabulate}{\hspace*{0.5cm}}
\begin{lstlisting}
\begin{tabular}{ | l | }
\hline
\\
int x = 3;\\
int y = 7;\\
double z = 4.25;\\
\\
\hline
\end{tabular}
\end{lstlisting}
\\\\
Размещается пустая строка между функциями и между группами утверждений.\\
Перед каждой открывающей скобкой '\{' ставьтся пробел, а после - перенос строки.
Также отступ увеличивается на один шаг после каждой открывающей скобки '\{', и уменьшается его на один шаг перед каждой закрывающей скобке '\}'. \\
Для отступов используется табуляция. Размер таба равен 4 пробелам.\\

\begin{lstlisting}
\begin{tabular}{ | l | }
\hline
\\
void first() \{\\
\Tabulate return 0;\\
\}\\
\\
void second() \{\\
\Tabulate return 0;\\

\}\\
\\
\hline
\end{tabular}
\end{lstlisting}

\subsection{Названия переменных, функций, структур и классов}
Функции и переменные имеют описательные имена, отражающие их содержание или назначение. Избегается однобуквенные названия, за исключением итераторов.\\
Имена функций и пересенных записываются, используя верблюжийРегистр. Название классов и структур - ПаскальныйРегистр, а констант — ВЕРХНИЙ\_РЕГИСТР.\\\\
\begin{lstlisting}
\begin{tabular}{ | l | }
\hline
\\
int maxNumber(...);\\\\
int maxNunber;\\\\
#define MAX\_NUMBER 100;\\\\
typedef struct MaxNumber \{\\
    \Tabulate ...\\
\} Node;
\\\\
\hline
\end{tabular}
\end{lstlisting}


\subsection{Ввод и вывод данных}
\begin{lstlisting}
\begin{tabular}{ | l | }
\hline
\\
std::cout \ll "Please, enter your number: ";\\
std::cin \gg a;\\
std::cout \ll "Your number " \ll a \ll std::endl;\\
\\
\hline
\end{tabular}
\end{lstlisting}
\newpage

\subsection{Циклы\cite{two}}\\
Используется один пробел между ключевым словом и открывающей скобкой:\\
До и после операторов сравнения используется отступ в один пробел:\\\\

\large \textbf{for}\\\\
\normalsize
\begin{lstlisting}
\begin{tabular}{ | l | }

\hline
\\
for (int i = 0; i < 10; i++) \{\\
         \Tabulate for (int j = 0; j < 5; j++) \{\\
               \Tabulate\Tabulate...; \\
           \Tabulate \} \\
\}
\\\\
\hline

\end{tabular}
\\\\
\end{lstlisting}
\\\large \textbf{while}\\\\
\normalsize
\begin{left}
\begin{lstlisting}
\begin{tabular}{ | l | }

\hline
\\
while (i < 10) \{\\
               \Tabulate...; \\
\}
\\\\
\hline

\end{tabular}
\\\\
\end{lstlisting}\\
\end{left}\\\\
\\\large \textbf{do while}\\\\
\begin{left}
\begin{lstlisting}
\begin{tabular}{ | l | }

\hline
\\
do \{\\
               \Tabulate...; \\
        \} while (i < 10);
\\\\
\hline

\end{tabular}
\\\\
\end{lstlisting}
\end{left}\



\newpage


\subsection{if else}
Каждый if и else пишется на отдельной строчки(кроме случая else if, тогда они пишутся через пробел). Фигурные скобки ставятся только в случае, если при попадании в условие выполняется более одного действия, иначе оно пишется на новой строке через отступ.\\
Используется один пробел между ключевым словом и открывающей скобкой:\\
До и после операторов сравнения используется отступ в один пробел:\\\\
\begin{lstlisting}
\begin{tabular}{ | l | }

\hline
\\
if (a >= b) \{\\
\Tabulate std::cout \ll a ;\\
\Tabulate return a;\\
\}\\
else if (b >= c) \\
\Tabulate return b;\\
else\\
\Tabulate return c;\\
\\
\hline

\end{tabular}
\\\\
\end{lstlisting}

\subsection{Функции}
В порядке написании функций стоит придерживаться последовательности применения, чтобы не было ошибок.\\\\
Между функциями обязательно оставляется пустая строка для более удобного листинга программы.\\\\
Закрывающая скобка всегда ставится в соответствии с уровнем начала функции, к которой она принадлежит\\\\
\begin{lstlisting}
\begin{tabular}{ | l | }

\hline
\\
void createStudent(int age, ...) \{\\
    \Tabulate ...\\
\}\\\\

void printStudent(void *student) \{\\
    \Tabulate ...\\
\}\\\\

int main() \{\\
	\Tabulate ...\\
	\\
	\Tabulate createStudent(age, ...);\\
	\Tabulate ...\\
	\\
	\Tabulate printStudent(student);\\

\}\\
\\
\hline

\end{tabular}
\\\\
\end{lstlisting}


\subsection{Структуры}
\begin{lstlisting}
\begin{tabular}{ | l | }

\hline
\\typedef struct Node \{\\
    \Tabulate string *line;\\
    \Tabulate int height;\\
    \Tabulate struct Node *left;\\
    \Tabulate struct Node *right;\\
\} Node;
\\\\
\hline

\end{tabular}
\\\\
\end{lstlisting}

\subsection{Классы}

\begin{lstlisting}
\begin{tabular}{ | l |}

\hline
\\

class Point \{\\
private:\\
   \Tabulate int x;\\
   \Tabulate int y;\\
public:\\
  \Tabulate  Point() \{\\
    \Tabulate\Tabulate    this->x = 0;\\
    \Tabulate\Tabulate    this->y = 0;\\
   \Tabulate \}\\
 \Tabulate   Point(int x, int y) \{\\
    \Tabulate\Tabulate    this->x = x;\\
    \Tabulate\Tabulate    this->y = y;\\
   \Tabulate \}\\
    \Tabulate  void setX(int x) \{\\
    \Tabulate\Tabulate    this->x = x;\\
   \Tabulate \}\\
   \Tabulate  void setY(int y) \{\\
    \Tabulate\Tabulate    this->y = y;\\
   \Tabulate \}\\
   \Tabulate  int getX() \{\\
    \Tabulate\Tabulate    return this->x;\\
   \Tabulate \}\\
   \Tabulate  int getY() \{\\
    \Tabulate\Tabulate    return this->y;\\
   \Tabulate \}\\
\};\\
\hline

\end{tabular}
\\\\
\newpage

\end{lstlisting}
\begin{thebibliography}{1}
\bibitem{CPP}
ISO/IEC 14882 Programming languages — C++.
\bibitem{one}CS 106B: Programming Abstractions (C++)
Summer 2015 \\ [http://stanford.edu/class/archive/cs/cs106b/cs106b.1158/styleguide.shtml]
\bibitem{two}Циклы. Операторы цикла. \\
[https://www.bestprog.net/ru/2017/09/04/cycles-operators-of-the-cycle-for-while-do-while\_ru/]
\bibitem{three}Объявления Typedef.\\
[https://docs.microsoft.com/ru-ru/cpp/c-language/typedef-declarations?view=msvc-16]
\end{thebibliography}
\bibliography{sample}

\end{document}